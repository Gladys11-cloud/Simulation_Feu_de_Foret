\section{Quelques concepts théoriques}

Dans cette partie, nous aborderons quelques concepts théoriques nécessaires à la compréhension de notre cas d'étude de simulation graphique d'un feu de forêt avec le langage Java. Nous commencerons par explorer le concept d'automate cellulaire, qui est une méthode de modélisation mathématique utilisée pour simuler des systèmes complexes tels que les feux de forêt. Ensuite, nous parlerons du diagramme de \textit{\textit{Gantt} }, un outil de gestion de projet qui nous permettra de planifier les différentes tâches à réaliser pour mener à bien notre projet. Nous aborderons également le langage de modélisation unifiée (UML) qui sera utilisé pour établir une modélisation informatique de notre projet. Nous verrons également l'importance de \textit{Git}, un système de contrôle de version qui facilite le travail collaboratif sur un projet. Enfin, nous étudierons le langage de programmation \textit{Java}, qui sera utilisé pour mettre en œuvre la simulation graphique du feu de forêt.

\subsection{Automate cellulaire}

Les automates cellulaires sont des modèles mathématiques permettant de simuler des phénomènes qui se déroulent dans des espaces discrets.\parencite{wuensche2000cellular} Ils sont largement utilisés dans la modélisation de systèmes complexes tels que la météorologie, l'écologie, la biologie et la physique. Dans notre cas d'étude, nous utiliserons les automates cellulaires pour simuler l'évolution d'un feu de forêt en fonction de différents paramètres tels que le vent, l'humidité, le type de végétation et le climat.

Le voisinage de \textit{Von Neumann} est l'une des méthodes les plus courantes pour la modélisation de l'environnement à travers les automates cellulaires.\parencite{franceschini2020cellular} Cette méthode suppose que chaque cellule est en contact avec ses quatre voisins immédiats : en haut, en bas, à gauche et à droite. La règle de transition est alors appliquée à chaque cellule en fonction de l'état de ses voisins. Pour notre simulation de feu de forêt, le voisinage de \textit{Von Neumann} nous permettra de prendre en compte l'influence des cellules environnantes sur le comportement du feu.

L'utilisation des automates cellulaires et du voisinage de \textit{Von Neumann} nous permettra de simuler de manière réaliste l'évolution d'un feu de forêt et d'analyser l'impact de différents facteurs tels que la direction du vent, l'humidité du sol, le type de végétation et le climat sur la propagation du feu.

\subsection{Diagramme de \textit{Gantt}}

Le diagramme de Gantt est un outil de gestion de projet qui permet de planifier les différentes étapes d'un projet et de suivre l'avancement des tâches. Il a été inventé au début du 20ème siècle par Henry Gantt pour planifier les activités de production et a depuis été adapté pour être utilisé dans une variété de projets et d'industries. 

Dans notre projet de simulation de feu de forêt, nous avons utilisé le diagramme de Gantt pour identifier les tâches à effectuer, estimer leur durée et les organiser dans une chronologie cohérente. Cela nous a permis de suivre l'avancement du projet, de respecter les délais et de détecter les retards potentiels.

Dans l’ensemble, le diagramme de Gantt a été un outil très utile pour la gestion de notre projet de simulation de feu de forêt. Nous recommandons son utilisation pour toute personne impliquée dans la gestion de projets, qu’ils soient personnels ou professionnels.

%\image{assets/images/1.bpmn.png}{Titre de la figure 1}{figure1}

\subsection{UML}

UML (Unified Modeling Language) est un langage visuel pour modéliser les systèmes logiciels. Il est utilisé dans l'industrie pour représenter graphiquement les modèles de systèmes logiciels. 

Différents types de diagrammes UML peuvent être créés pour représenter différents aspects d'un système. Le diagramme de classes est un type de diagramme qui représente les classes et leurs relations dans un système. Les classes sont représentées par des rectangles contenant leur nom, attributs et méthodes, et les relations sont représentées par des flèches. Les flèches peuvent être unidirectionnelles ou bidirectionnelles. 

Le diagramme de classes est un outil essentiel pour décrire la structure d'un système orienté objet. Ce qui permet aux membres de l'équipe de se représenter mentalement le fonctionnement général du programme et d'avoir une meilleur perception des résultats attendus. 


\subsection{Git}

Git est un système de contrôle de version très populaire pour le développement de logiciels. Il permet à plusieurs développeurs de travailler ensemble sur un même projet tout en conservant un historique de toutes les modifications apportées au code.

Git utilise une approche décentralisée, dans laquelle chaque développeur possède une copie locale du code source et peut partager les modifications apportées via un référentiel Git centralisé. Git permet de fusionner facilement les modifications apportées par différents développeurs, ce qui facilite la collaboration. 

Nous avons utilisé Git pour notre projet de simulation de feu de forêt afin de gérer le code source et faciliter la collaboration entre les membres de l'équipe. Nous avons créé un référentiel Git centralisé et utilisé les commandes Git de base pour gérer les modifications apportées au code source. 

En utilisant Git, nous avons pu conserver un historique de toutes les modifications apportées au code source et travailler simultanément sur différentes fonctionnalités sans risquer de perdre des modifications importantes.

\subsection{Java}

Java est un langage de programmation orienté objet très populaire dans le développement d'applications pour diverses plates-formes, grâce à sa portabilité. Il est souvent utilisé pour développer des applications de bureau, des applications Web et des jeux vidéo. 

Java dispose également d'une grande communauté de développeurs et de nombreuses bibliothèques open source qui facilitent le développement d'applications.

Pour la simulation de feu de forêt, Java est un choix judicieux car il peut être utilisé pour créer une interface graphique conviviale pour afficher la simulation en temps réel. Le langage est également suffisamment puissant pour permettre une manipulation facile des automates cellulaires et des calculs mathématiques nécessaires pour la simulation.