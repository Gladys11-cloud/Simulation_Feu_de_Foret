\section{Revue de la littérature }

La revue de la littérature effectuée dans le cadre de ce rapport a permis de recenser plusieurs études portant sur la modélisation de la propagation des incendies de forêt à l'aide des automates cellulaires. Parmi ces études, nous avons identifié neuf articles scientifiques qui ont retenu notre attention. Ces études présentent différentes approches et méthodologies pour la modélisation de la propagation des incendies de forêt, en utilisant les automates cellulaires comme outil de simulation. Les articles identifiés incluent : Hernandez Encinas et al. (2006) \parencite{hernandez2006}, Karafyllidis et Thanailakis (1996) \parencite{ioannis1996}, Alexandridis et al. (2008) \parencite{alexandridis2008}, Berjak et Hearne (2001) \parencite{berjak2001}, Drossel et Schwabl (1992) \parencite{drossel1992}, Ghisu et al. (2015) \parencite{ghisu2015}, Mutthulakshmia et al. (2020) \parencite{mutthulakshmia2020}, Freire et DaCamara (2019) \parencite{freire2019} et Hernandez Encinas et al. (2006) \parencite{encinas2006}.

L'étude de Hernandez Encinas et al. (2006) \parencite{hernandez2006} présente un modèle de propagation des incendies forestiers utilisant des automates cellulaires hexagonaux. Les auteurs ont intégré des paramètres tels que la densité des arbres, la vitesse et la direction du vent, ainsi que la présence de routes et de cours d'eau pour simuler la propagation des incendies. Les résultats montrent une corrélation significative entre les conditions environnementales et la vitesse de propagation de l'incendie.

L'étude de Karafyllidis et Thanailakis (1996) \parencite{ioannis1996} a proposé un modèle de prédiction de la propagation des incendies forestiers en utilisant des automates cellulaires à deux dimensions. Le modèle a été développé en tenant compte des caractéristiques de la végétation, des conditions météorologiques et topographiques, ainsi que des comportements humains. Les résultats ont montré une corrélation significative entre la vitesse de propagation des incendies et la densité de la végétation. Cette étude a utilisé un voisinage de Moore pour la propagation de l'incendie.

L'étude de Alexandridis et al. (2008) \parencite{alexandridis2008} a présenté un modèle de prédiction de la propagation des incendies forestiers à l'aide d'automates cellulaires à deux dimensions. Le modèle a été développé pour simuler l'incendie qui a balayé l'île de Spetses en 1990. Les auteurs ont inclus des paramètres tels que la densité de la végétation, la direction et la vitesse du vent, ainsi que l'influence des facteurs topographiques. Les résultats ont montré une corrélation significative entre la vitesse de propagation de l'incendie et la densité de la végétation ainsi que la direction et la vitesse du vent.

L'étude de Berjak et Hearne (2001) \parencite{berjak2001} a présenté un modèle d'automate cellulaire amélioré pour simuler les incendies dans un système de savane spatialement hétérogène. Les auteurs ont pris en compte les effets des paramètres tels que la densité de la végétation, la direction et la vitesse du vent, la densité du combustible et la topographie. Les résultats ont montré une corrélation significative entre la vitesse de propagation de l'incendie et la densité de la végétation ainsi que les conditions météorologiques. Cette étude a utilisé un voisinage de von Neumann pour la propagation de l'incendie.

L'article de Drossel et Schwabl (1992) \parencite{drossel1992} introduit un modèle de propagation d'incendie basé sur l'idée de la critique auto-organisée (self-organized criticality). Ce modèle suppose que les conditions de départ, comme la densité des arbres ou la présence d'herbes sèches, sont distribuées de manière aléatoire, ce qui conduit à une propagation d'incendie auto-organisée, c'est-à-dire que la propagation suit une loi de puissance plutôt que d'être régulière. Les auteurs ont également examiné la distribution spatiale de la taille des incendies, qui suit également une loi de puissance. Bien que ce modèle soit assez différent de celui que nous avons utilisé dans notre étude, il met en évidence l'importance des facteurs aléatoires et de la complexité des interactions dans la propagation des incendies.

Ghisu et al. (2015) \parencite{ghisu2015} ont proposé quant à eux un algorithme optimisé pour la simulation de la propagation des incendies de forêt en utilisant un automate cellulaire testé sur un ensemble de données expérimentales recueillies sur le terrain. Les résultats ont montré que la simulation basée sur l'algorithme proposé est capable de reproduire la propagation réelle du feu avec une bonne précision. Les auteurs ont également comparé leur méthode avec d'autres approches d'automates cellulaires existantes et ont montré que leur algorithme est plus efficace et plus précis. Ce travail met en avant l'importance de l'optimisation des algorithmes pour la simulation de la propagation des incendies de forêt, ce qui peut être utile pour la gestion et la prévention des incendies de forêt. De plus, la méthodologie utilisée dans cet article peut être applicable à d'autres domaines qui utilisent des automates cellulaires pour simuler des phénomènes naturels.

L'étude de Mutthulakshmia et al. (2020) \parencite{mutthulakshmia2020} propose une simulation de la propagation des feux de forêt à l'aide d'automates cellulaires basé sur des règles qui décrivent l'ignition, la propagation et l'extinction des feux. Ils ont également incorporé des règles pour la lutte contre les incendies, telles que l'utilisation de coupe-feu, d'eau et de produits chimiques. La simulation a été réalisée sur une zone de 1000x1000 cellules et les résultats ont été validés à l'aide de données satellitaires réelles. Les résultats ont montré que le modèle proposé est capable de reproduire avec précision les schémas de propagation du feu observés dans la réalité. Cette étude est intéressante car elle montre l'importance de prendre en compte des facteurs externes dans la modélisation des feux de forêt, tout en mettant en avant les avantages des automates cellulaires pour ce type de simulation.

Enfin, Les études menées respectivement par Freire et DaCamara (2019) \parencite{freire2019}, et par Hernandez Encinas et al. (2006) \parencite{encinas2006} utilisent elles aussi les automates cellulaires pour simuler la propagation des incendies de forêt. La première utilise une méthode de modélisation basée sur la propagation de la chaleur et sur la détermination des zones critiques. Les auteurs ont testé leur modèle en le comparant à des données réelles de feux de forêt et ont constaté que le modèle avait une précision satisfaisante. La deuxième étude quant à elle, utilise une approche similaire à celle de Hernandez Encinas et al. (2006) \parencite{hernandez2006} en utilisant un voisinage de Von Neumann pour déterminer la probabilité de propagation de l'incendie de forêt. Les auteurs ont également proposé un algorithme optimisé pour améliorer la précision de la simulation.

Ces études ont toutes pour point commun l'utilisation des automates cellulaires pour modéliser la propagation des incendies de forêt. Elles diffèrent toutefois dans les choix qu'elles font quant aux règles de propagation du feu. Certaines d'entre elles, telles que l'étude de Drossel et Schwabl (1992) \parencite{drossel1992}, utilisent des règles relativement simples pour modéliser la propagation du feu, tandis que d'autres, comme l'étude de Ghisu et al. (2015) \parencite{ghisu2015}, ont recours à des règles plus complexes pour prendre en compte différents facteurs environnementaux.

Notre choix de nous concentrer sur le voisinage de Von Neumann et la prise en compte de facteurs externes dans le calcul de la probabilité de propagation est directement lié aux résultats et approches présentés dans ces études. En effet, plusieurs de ces études ont utilisé des approches similaires pour modéliser la propagation des incendies de forêt, en mettant l'accent sur l'influence des facteurs environnementaux et de la topographie dans la propagation du feu. Nous avons donc cherché à nous appuyer sur ces résultats pour réaliser une implémentation graphique d’un modèle simplifié de propagation des incendies de forêt à l'aide des automates cellulaires.