\section{Cas d’étude : Simulation graphique d’un feu de forêt avec le langage Java}

Dans cette partie, nous allons étudier un cas pratique de simulation graphique d'un feu de forêt en utilisant le langage de programmation Java. La simulation est basée sur des automates cellulaires et prend en compte plusieurs paramètres tels que le voisinage de \textit{Von Neumann}, la direction du vent, l'humidité du sol, le type de végétation, et les conditions climatiques. Nous allons explorer la modélisation mathématique du problème, la gestion de projet, la modélisation informatique, la conception de l'interface utilisateur et la mise en œuvre de la solution. Cette étude permettra de mieux comprendre l'application des automates cellulaires dans la simulation de phénomènes naturels et de développer des compétences en programmation Java.

\subsection{Modélisation mathématique du problème}
\label{sec:modélisationMath}

La modélisation mathématique est une étape importante pour comprendre les mécanismes de la propagation du feu dans une forêt et prédire son évolution. Nous avons utilisé une grille de cellules pour représenter la forêt, où chaque cellule peut avoir trois états : sol nu, arbre vivant et arbre en feu. Nous avons défini des règles de propagation qui prennent en compte divers facteurs tels que le {\it voisinage}, la {\it direction du vent}, la {\it saison}, l'{\it humidité du sol} et le {\it type de végétation}. Ces règles ont été implémentées dans un modèle mathématique qui nous permet de simuler la propagation du feu dans la forêt et d'étudier les effets de chaque facteur sur la propagation de l'incendie.

La modélisation de la forêt a été effectuée à l'aide d'une grille de taille $n \times n$ contenant $n^2$ cellules. Chaque cellule peut avoir quatre états possibles tels que décrit dans le tableau \ref{tab:typesCellules}. Des règles de propagation ont été établies afin de déterminer l'état futur de chaque cellule en fonction de son état actuel et de l'état de ses voisins. Si la cellule est un arbre en feu, son état futur est "arbre en cendre". Si elle est un arbre vivant et qu'aucun de ses voisins n'est en feu, alors son état futur est le même que son état actuel. Si elle est un arbre vivant et au moins un de ses voisins est en feu, alors sa probabilité de prendre feu est calculée. La probabilité initiale est fixée à $0.6$, ce qui correspond à une probabilité raisonnable pour un incendie de forêt.

\tableau{Types de cellules}{typesCellules}{|c|c|c|}{%
    \ligneTableau{\textbf{Type de cellule} & \textbf{Couleur de la cellule} & \textbf{Code}}
    \ligneTableau{Sol nu & Ivoire & 0}
    \ligneTableau{Arbre vivant & Vert & 1}
    \ligneTableau{Arbre en feu & Rouge & 2}
    \ligneTableau{Arbre en cendre & Gris & 3}
}

Cependant, cette probabilité fixe peut ne pas être réaliste dans toutes les situations. C'est pourquoi des facteurs externes ont été intégrés dans le modèle afin d'améliorer sa précision. Ces facteurs incluent le voisinage, la direction du vent, la saison, l'humidité du sol et le type de végétation.

\subsubsection{Impact du voisinage}

Dans notre modèle de simulation de la propagation d'un incendie forestier, le voisinage d'une cellule est un élément important qui peut avoir un impact significatif sur l'état de la cellule elle-même. Le voisinage de \textit{Von Neumann} a été choisi pour notre simulation, car il permet de prendre en compte les cellules voisines situées dans les quatre directions cardinales (nord, sud, est, ouest). Le voisinage de \textit{Von Neumann} est défini comme l'ensemble des cellules situées à une distance de \textit{Manhattan} égale à 1 de la cellule en question. Autrement dit, si nous considérons une cellule donnée dans notre grille, son voisinage de \textit{Von Neumann} est composé de ses quatre voisins directs situés au nord, au sud, à l'est et à l'ouest. La figure \ref{fig:VoisinageNeumann} présente de façon graphique ce voisinage.

\image{assets/images/Nbhd_neumann.png}{Voisinage de Von Neumann}{VoisinageNeumann}{0.3}

Lorsqu'une cellule vivante a au moins un voisin en feu, nous calculons la probabilité pour qu'elle aussi s'enflamme. Partant de la probabilité de base p initialement fixée à 0.6, la formule \ref{eq:prob_voisinage} est appliquée pour mettre à jour celle-ci de sorte qu’elle prenne en compte le nombre n de voisins en feu de la cellule.
\begin{equation}
    p = p + (n \times 0.05)
    \label{eq:prob_voisinage}
\end{equation}

Le voisinage de Von Neumann peut avoir un impact significatif sur la simulation de la propagation de l'incendie forestier, car si plusieurs cellules sont en feu dans le voisinage direct d'une cellule donnée, la probabilité pour qu'elle s'enflamme à son tour augmente considérablement. Cependant, si la cellule a des voisins qui ne sont pas en feu, cela peut diminuer la probabilité qu'elle aussi s'enflamme. En prenant en compte le voisinage de Von Neumann dans notre simulation, nous pouvons avoir une représentation plus réaliste de la propagation de l'incendie forestier. Cela nous permet également d'explorer les différents facteurs qui peuvent influencer cette propagation et d'ajuster notre modèle en conséquence.

\subsubsection{Impact de la direction du vent}

La direction du vent est un facteur important dans la propagation du feu en forêt. Dans notre modélisation, nous avons pris en compte la direction du vent pour déterminer la probabilité qu'une cellule vive prenne feu en fonction de ses voisins. Le vent peut être défini par rapport aux quatre points cardinaux et peut influencer la propagation du feu de différentes manières. Si le vent souffle dans la direction opposée à celle d'une cellule en feu, la probabilité de propagation diminue car le vent peut empêcher la propagation en poussant la fumée et les flammes dans la direction opposée. En revanche, si le vent souffle dans la même direction que celle d'une cellule en feu, la probabilité de propagation augmente car le vent peut alimenter le feu en oxygène. 

Dans notre modélisation, nous avons utilisé une valeur binaire pour indiquer si le vent pousse le feu vers la cellule en question. Si la valeur est 1, la probabilité que la cellule prenne feu augmente de 0,5, tandis que si la valeur est 0, la probabilité diminue de 0,1. La formule \ref{eq:prob_vent} résume cet impact.

\begin{equation}
    p = \begin{cases}
    p + 0.5 & \text{si } b = 1 \\
    p - 0.1 & \text{si } b = 0
    \end{cases}
    \label{eq:prob_vent}
\end{equation}

L'utilisateur peut choisir \textit{INDIFFERENT} comme direction du vent pour ignorer son impact. Dans ce cas, la probabilité de propagation dépendra uniquement de l'état des voisins de la cellule en question, sans considérer la direction du vent. 

En somme, la direction du vent est un facteur important à prendre en compte pour modéliser la propagation du feu dans la forêt, et nous avons tenu compte de cet impact dans notre modèle en ajustant la probabilité de propagation en fonction de la direction du vent et de la position des cellules en feu, créant ainsi une simulation plus réaliste.

\subsubsection{ Impact du climat}

Les saisons ont un impact important sur la propagation des feux de forêt, car les variations de température et d'humidité peuvent considérablement changer la probabilité de propagation d'un feu dans une forêt. 

Dans notre modèle, nous avons donc pris en compte cet impact en modifiant la probabilité initiale pour chaque cellule. En hiver, la probabilité initiale de prendre feu a été réduite de 10\% pour les arbres, car les températures sont basses et l'humidité est souvent plus élevée. En été, la probabilité initiale de prendre feu a été augmentée de 20\% pour les arbres, car les températures sont élevées et l'humidité est généralement plus faible. Pour le printemps et l'automne, nous avons considéré que les conditions étaient relativement neutres et n'avaient donc pas d'impact significatif sur la probabilité de propagation d'un feu. 

En incluant l'impact des saisons sur la probabilité initiale de prendre feu, notre modèle est plus réaliste et peut être utilisé pour explorer différents scénarios pour les feux de forêt dans différentes saisons, ce qui est utile pour la planification des mesures de prévention et de lutte contre les feux de forêt.

\subsubsection{Impact de l’humidité du sol}

L'humidité du sol peut ralentir ou éteindre les incendies de forêt, car l'eau absorbe la chaleur et réduit la température autour des arbres en feu.

Dans notre modèle, nous avons réduit la probabilité qu'un arbre prenne feu lorsque le sol est humide. Cela peut avoir un impact significatif sur la propagation des incendies, en fonction des conditions météorologiques et environnementales locales. Si le sol est humide, cela peut aider à ralentir la propagation des flammes, tandis que si le sol est sec, cela peut aggraver la situation. 

En conclusion, l'humidité du sol doit être prise en compte dans la modélisation des incendies de forêt pour mieux simuler leur propagation.

\subsubsection{Type de végétation}

Le type de végétation est un facteur important qui peut affecter la propagation du feu dans une forêt. Certaines espèces d'arbres sont plus résistantes au feu que d'autres, par exemple les conifères sont plus inflammables que les feuillus. 

Dans notre modèle, nous avons ajouté une variable booléenne \textit{arbre peu inflammable} pour représenter la résistance au feu des arbres. Si cette variable est vraie, alors la probabilité de propagation du feu est réduite de 20\%. Nous avons également pris en compte l'impact de l'humidité du sol sur la résistance au feu des arbres, car les arbres dans des zones humides ou avec un accès à de l'eau souterraine sont souvent plus résistants aux incendies que ceux dans des régions arides. 

En modifiant la résistance au feu des arbres, notre modèle peut simuler différents types de forêts et étudier leur comportement face aux incendies, ce qui aidera à élaborer des stratégies de prévention et de lutte contre les incendies.

\subsection{Gestion de projet}

La gestion de projet est un aspect essentiel de tout travail collaboratif. Dans notre projet de modélisation d'incendies de forêt, nous avons utilisé plusieurs outils de gestion de projet pour nous aider à rester organisés, à suivre les progrès et à gérer les tâches de manière efficace. Les quatre technologies que nous avons utilisées sont le diagramme de \textit{Gantt}, \textit{Trello}, \textit{Git} et la méthodologie \textit{agile}.

\subsubsection{Diagramme de Gantt}

Le diagramme de Gantt est un outil de gestion de projet utilisé pour planifier et suivre les tâches à accomplir dans un projet. Il permet de visualiser les tâches, leur durée et leur dépendance les unes par rapport aux autres. Cet outil a été très utile pour notre projet car il nous a permis de suivre le progrès et de voir les tâches restantes à accomplir. Nous avons pu identifier les tâches qui prenaient plus de temps que prévu et ajuster notre planification en conséquence. Le diagramme de Gantt a également aidé à définir les rôles et les responsabilités de chaque membre de l'équipe. La figure \ref{fig:DiagrammeDeGantt} présente une capture ponctuelle du diagramme de Gantt que nous avons utilisé.

\image{assets/images/Diagramme_de_gantt.jpg}{Diagramme de Gantt du projet}{DiagrammeDeGantt}{1}

\subsubsection{Trello}

Trello est un outil de gestion de projet avec des tableaux kanban. Il permet de créer des listes de tâches, des cartes pour chaque tâche, et des étiquettes pour classer les tâches. Nous avons utilisé Trello pour suivre les tâches quotidiennes et garder une trace des progrès. Nous avons pu ajouter des commentaires et des notes pour que chaque membre de l'équipe reste informé de l'avancement. Voici une capture d'écran de notre tableau Trello montrant les différentes tâches et leurs statuts.

\subsubsection{Git}

Git est un outil de gestion de version pour le code source. Il permet de suivre les modifications du code, de restaurer les versions précédentes, et de collaborer sur le code avec plusieurs membres de l'équipe. Nous avons utilisé Git pour notre code de modélisation d'incendies de forêt \parencite{gitrepo}, ce qui a facilité la collaboration entre les membres de l'équipe. Nous avons pu travailler sur différentes parties du code en même temps sans craindre de perdre des modifications importantes. Nous avons également utilisé les branches de Git pour développer de nouvelles fonctionnalités sans interrompre le développement actuel. Plus concrètement, les branches suivantes ont été créée :

\begin{enumerate}
    \item Une branche \textit{master} pour la version finale du code
    \item Une branche test pour les fonctionnalités prêtes à être validées par les membres de l’équipe
    \item Une branche dev pour les fonctionnalités consolidées mais qui doivent être intégrées au projet dans son ensemble
    \item Une branche pour chaque membre d’équipe pour le développement proprement dit des tâches.
\end{enumerate}

La figure \ref{fig:BranchesGit} présente une représentation graphique de la stratégie de branches que nous avons adopté ainsi qu'un aperçue ponctuel de l’historique des commit.

\image{assets/images/Strategie_branches_et_historique_commit.png}{Stratégie des branches et historique des commit}{BranchesGit}{1}

\subsubsection{Méthodologie Agile}

Nous avons utilisé la méthodologie agile pour gérer notre projet de modélisation d'incendie de forêt. 

Cette méthode implique des itérations courtes appelées sprints, qui durent entre une et quatre semaines. Chaque sprint a un objectif spécifique et des tâches sont assignées aux membres de l'équipe pour atteindre cet objectif.

Cette méthode permet une collaboration étroite avec les parties prenantes, une meilleure visibilité sur la progression du projet et encourage l'amélioration continue. En fin de compte, la méthodologie agile a grandement amélioré notre efficacité en tant qu'équipe.

\subsection{Modélisation informatique du problème}

La modélisation informatique est une étape cruciale dans la conception d'un logiciel. Elle permet de représenter les différentes composantes du système ainsi que les interactions entre elles. Dans ce projet, nous avons utilisé le diagramme de classe de l'UML pour modéliser notre système de simulation de feu de forêt. Ce diagramme nous a permis de visualiser les différentes classes du système, les relations entre elles ainsi que les attributs et les méthodes de chaque classe. De plus, cette modélisation nous a permis de mieux comprendre le fonctionnement de notre système et ainsi de faciliter la phase d'implémentation.

\subsubsection{Construction et amélioration du diagramme de classes}

La construction du diagramme de classe UML a été une étape importante pour modéliser notre projet de simulation d'incendie de forêt. Chacun des membres de l'équipe a travaillé sur un diagramme de classes individuel pour représenter sa vision du projet. Après discussion et prise en compte des meilleures idées, nous avons sélectionné la version initiale du diagramme de classes qui était la plus complète et intuitive. Cette version initiale a servi de base pour l'implémentation de notre programme.

Cependant, nous avons réalisé qu'il était possible de simplifier le diagramme et de le rendre encore plus efficace en termes de qualité de code. Nous avons donc amélioré le diagramme en prenant en compte les principes SOLID de programmation pour garantir une structure simple et claire. Nous avons regroupé les différentes classes d'arbres en une seule classe avec un attribut état pour simplifier le diagramme.

La version finale du diagramme de classes a été adoptée par l'équipe pour obtenir une structure plus simple et claire pour le projet. Cela nous a permis d'obtenir un code plus facile à comprendre et à implémenter.

\image{assets/images/version_finale_diagramme_de_classes.png}{Version finale du diagramme de classes}{umlFinal}{0.7}

\subsubsection{Description du diagramme de classes}

Le diagramme de classes final comprend quatre classes principales et deux autres qui en spécialisent une. Ces classes sont les suivantes :

1. Classe \texttt{App}: Cette classe est la fenêtre principale du projet, point de départ de l’application. Elle contient la grille de simulation et le menu, et gère la simulation. Elle possède plusieurs attributs tels que la grille, le menu, la densité des arbres, la direction du vent, la saison, le taux d'arbres peu inflammables, le taux de cellules humides, le taux de forêt déjà brûlée et un booléen qui permet de savoir si l'utilisateur a cliqué sur le bouton \textit{Arrêter}. Elle a également des méthodes qui permettent de générer la grille et de lancer la simulation de façon itérative jusqu’à ce qu’il n’y ait plus de changement dans la grille ou que l’utilisateur clique sur le bouton \textit{Arrêter}.

2. Classe \texttt{Menu}: Cette classe gère les interactions de l'utilisateur avec l'interface graphique. Elle contient les boutons, les champs de texte et les lignes de séparation qui permettent de configurer la grille et de lancer la simulation. Elle a des attributs tels que la largeur et la hauteur minimale du panneau, ainsi que des couleurs en hexadécimal pour les boutons et les champs de texte.

3. Classe \texttt{Grille}: Cette classe implémente la grille de simulation et permet de la mettre à jour. Elle a des attributs tels que le nombre de lignes et de colonnes de la grille, ainsi qu'un tableau à deux dimensions contenant toutes les cellules de la grille. Elle a également des méthodes qui permettent de construire la grille et de mettre à jour la grille lors de la simulation. Elle possède des méthodes utiles permettant d’accéder à une cellule spécifique de la grille, d’en déterminer les voisins et même de la modifier.

4. Classe \texttt{Cellule}: Cette classe abstraite représente les différentes composantes de la forêt, telles que le sol nu, l'arbre vivant, l'arbre en feu et l'arbre en cendres. Elle a une méthode abstraite, \texttt{calculeEtatFutur()}, qui permet de calculer l'état futur d'une cellule en se basant sur plusieurs critères tels que présentés dans la section \ref{sec:modélisationMath}. Les classes \texttt{SolNu} et \texttt{Arbre} héritent de cette classe.

5. Classe \texttt{SolNu}: Cette classe hérite de la classe \texttt{Cellule} et permet de modéliser un sol nu, qui correspond à une parcelle de terrain sans arbre. Sa méthode principale est \texttt{calculeEtatFutur()}, qui retourne simplement l'état courant de la cellule car un sol nu ne change pas d'état après le passage du feu.

6. Classe \texttt{Arbre}: Cette classe hérite également de la classe \texttt{Cellule} et permet de modéliser un arbre. Ses méthodes permettent de construire des arbres ayant des caractéristiques différentes afin de pouvoir observer le comportement du feu faces à divers types de forêts. Cette classe implémente également la méthode \texttt{calculeEtatFutur()}.

\subsection{Design de l’interface graphique}

Le design de l'interface graphique est une étape cruciale du développement d'un logiciel car il permet de concevoir visuellement la façon dont les utilisateurs vont interagir avec l'application. 

Pour notre simulateur d'incendie de forêt, nous avons utilisé l'outil \textit{Figma} pour concevoir l'interface graphique. La grille représente la forêt et est subdivisée en cellules, chacune représentant un arbre ou un sol nu. Le menu de configuration est subdivisé en quatre parties principales, permettant à l'utilisateur de configurer la forêt, les facteurs externes, et la simulation. L'interface graphique est conçue de manière claire et logique, ce qui permet à l'utilisateur de comprendre rapidement les différentes fonctionnalités et de les utiliser efficacement. 

La figure \ref{fig:designFigma} présente le design préalable tandis que la figure \ref{fig:interfaceRélle} présente l'interface effectivement implémentée.

\subsection{Mise en œuvre de la solution à l’aide du langage Java}

Cette partie présente trois algorithmes essentiels pour le bon fonctionnement de la simulation : la boucle de simulation, le calcul de l'état futur d'une cellule et le calcul de la probabilité de propagation du feu sur une cellule. Nous allons décrire ces algorithmes en détail pour mieux comprendre leur fonctionnement et leur rôle dans la simulation.

\subsubsection{Algorithme de la boucle de simulation}

L'algorithme de la boucle de simulation de la figure \ref{fig:boucleSimulation} est l'un des éléments clés de la solution. Cet algorithme est responsable de parcourir toutes les cellules de la grille et de calculer leur état futur en fonction de la direction du vent et de la saison. La méthode renvoie un tableau de deux entiers, le premier étant 0 si la simulation doit s'arrêter (s'il n'y a plus d'arbres en feu) et 1 sinon. Le deuxième entier représente le pourcentage de la forêt déjà brûlée.

La boucle de simulation commence par initialiser trois variables : continuerSimulation, nombreTotalArbre et nombreArbreEnCendre. La variable continuerSimulation est initialement à 0, mais elle sera mise à 1 si une cellule en feu est trouvée. Les variables nombreTotalArbre et nombreArbreEnCendre sont utilisées pour calculer le pourcentage de la forêt déjà brûlée. La boucle itère ensuite sur toutes les cellules de la grille. Pour chaque cellule, la méthode calculeEtatFutur est appelée, ce qui calcule l'état futur de la cellule en fonction de la direction du vent et de la saison. Si l'état futur de la cellule est 2 (en feu), la variable continuerSimulation est mise à 1. Enfin, les variables nombreTotalArbre et nombreArbreEnCendre sont mises à jour en fonction de l'état actuel et futur de la cellule. Le pourcentage de la forêt déjà brûlée est calculé en divisant le nombre d'arbres en cendres par le nombre total d'arbres et en multipliant par 100. Le tableau de deux entiers est ensuite retourné.

L'algorithme de la boucle de simulation est un élément essentiel de la solution car il permet de simuler la propagation du feu dans la forêt en calculant l'état futur de chaque cellule. Cet algorithme, ainsi que les autres algorithmes utilisés dans la solution, ont été implémentés en Java.

\subsubsection{Algorithme de calcul de l’état futur d’une cellule}
\label{sec:calculEtatFutur}

Cette méthode visible sur la figure \ref{fig:calculEtatFutur} utilise plusieurs critères pour déterminer l'état futur d'une cellule : le voisinage, la direction du vent, la saison, l'humidité du sol et le type de végétation.

Tout d'abord, si la cellule est un arbre déjà en cendre, son état futur reste le même. Si la cellule est un arbre en feu, son état futur est défini comme étant en cendre. Si la cellule est un arbre vivant et qu'aucun de ses voisins n'est en feu, elle reste dans son état actuel. Cependant, si la cellule est un arbre vivant et qu'au moins un de ses voisins est en feu, un calcul de probabilité est effectué pour déterminer si le feu va se propager à cette cellule. La probabilité de propagation du feu vers la cellule est calculée à l'aide de la méthode \texttt{calculeP()} qui prend en compte la direction du vent, la saison, l'humidité du sol et le type de végétation. Cette probabilité est ensuite utilisée pour faire passer le cas échéant la cellule à l'état en feu.

\subsubsection{Algorithme de calcul de la probabilité de propagation}

L'algorithme de calcul de la probabilité de propagation du feu présenté sur la figure \ref{fig:calculProbaPropagation} est relativement simple. La méthode prend en entrée la direction du vent et la saison courante et calcule la probabilité que l'arbre prenne feu en se basant sur plusieurs critères.

Tout d'abord, la méthode prend en compte l'impact du voisinage sur la probabilité \textit{p}. Si plusieurs cellules voisines sont en feu, alors la probabilité que l'arbre prenne feu sera plus grande. Ainsi, pour chaque voisin en feu, la probabilité p est augmentée de 0.05. Ensuite, la méthode prend en compte l'impact de la direction du vent sur la probabilité p. Si le vent souffle dans la direction de la cellule, la probabilité que l'arbre prenne feu sera plus grande. Ainsi, si le vent souffle dans la direction de la cellule, la probabilité p est augmentée de 0.5. Si le vent souffle dans une autre direction, la probabilité p est diminuée de 0.1.

La méthode prend également en compte l'impact de la saison sur la probabilité p. Si la saison est l'été, alors la probabilité que l'arbre prenne feu sera plus grande. Ainsi, si la saison est l'été, la probabilité p est augmentée de 0.2. Si la saison est l'hiver, la probabilité p est diminuée de 0.1. Ensuite, la méthode prend en compte l'impact de l'humidité du sol sur la probabilité p. Si le sol est humide, alors la probabilité que l'arbre prenne feu sera plus faible. Ainsi, si le sol est humide, la probabilité p est diminuée de 0.1. Enfin, la méthode prend en compte l'impact du type de végétation sur la probabilité p. Si l'arbre est peu inflammable, alors la probabilité que l'arbre prenne feu sera plus faible. Ainsi, si l'arbre est peu inflammable, la probabilité p est diminuée de 0.2.

En sortie, la méthode retourne la probabilité p, qui est comprise entre 0 et 1.