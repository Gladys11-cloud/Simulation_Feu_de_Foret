\pagenumbering{arabic}
\setcounter{page}{1}
\phantomsection
\addcontentsline{toc}{section}{Introduction}
\section*{Introduction}

L'augmentation régulière de la surface des forêts dévastées par les feux ces dernières années met en lumière l'urgence d'apporter des solutions pour mieux prévenir et gérer ces catastrophes naturelles. En France, l'année 2022 par exemple a été marquée par un record de surface végétative brûlée, avec plus de 62 000 hectares de forêt détruits lors de vagues de chaleur estivales. Face à l'urgence de ce phénomène qui ne cesse de s'aggraver chaque année, il est primordial de disposer d'outils permettant de mieux comprendre et maîtriser la propagation des feux de forêt. 

Dans ce contexte, la simulation de la propagation des incendies de forêt peut offrir une solution prometteuse pour mieux comprendre et anticiper l'expansion de ces feux. Ce projet a pour objectif de proposer une interface graphique en langage Java pour simuler la propagation d'un feu de forêt en fonction des règles définies dans un automate cellulaire. L'interface permettra à l'utilisateur de configurer la parcelle de forêt et les positions initiales des arbres en feu, puis de visualiser l'évolution de l'incendie selon les règles de transition de l'automate cellulaire.  

Ce rapport détaillera l'ensemble des étapes nécessaires à la réalisation de cette simulation, en passant en revue les concepts théoriques des automates cellulaires, le voisinage de Von Neumann, la modélisation mathématique et informatique du problème, ainsi que la gestion de projet et la conception de l'interface graphique. Nous explorerons également l'impact de différents facteurs sur la propagation du feu, tels que la direction du vent, l'humidité du sol, le climat et le type de végétation. Enfin, nous discuterons des différentes améliorations possibles de notre simulation, en explorant d’autres facteurs qui influencent le feu de forêt et en proposant des pistes pour intégrer des technologies plus avancées, telles que l'intelligence artificielle.