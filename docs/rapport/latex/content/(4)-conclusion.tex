\phantomsection
\addcontentsline{toc}{section}{Conclusion}
\section*{Conclusion}

En conclusion, ce projet de simulation de feu de forêt a permis de mettre en place un modèle simple mais réaliste de propagation du feu, en se basant sur plusieurs critères tels que la direction du vent, la saison, l'humidité du sol et le type de végétation. Le modèle a été implémenté en utilisant le langage de programmation Java et en se basant sur le principe de la programmation orientée objet. 

Nous avons vu comment l'algorithme de calcul de l'état futur d'une cellule ainsi que celui de la probabilité de propagation du feu ont été développés. Ces deux algorithmes sont les pierres angulaires de la simulation de feu de forêt et sont essentiels pour comprendre comment le feu se propage et comment il peut être maîtrisé. 

Il est important de noter que d'autres facteurs peuvent également influencer la propagation du feu de forêt, tels que la topographie, la densité de la forêt, la distance entre les arbres, la présence d'obstacles tels que des rivières ou des routes, et bien d'autres encore. Ces facteurs peuvent être intégrés dans la simulation pour en augmenter la précision. 

En outre, il serait possible d'intégrer des techniques d'intelligence artificielle pour prédire la direction du vent en se basant sur les données des feux de forêt passés. Cela pourrait permettre de mieux prévoir la propagation du feu et d'adopter des mesures de prévention plus efficaces. 

Enfin, ce projet de simulation de feu de forêt est un exemple de la façon dont la programmation peut être utilisée pour simuler des phénomènes naturels complexes et pour aider à mieux comprendre les mécanismes qui les sous-tendent. Il est donc possible d'explorer d'autres domaines en utilisant cette approche et de créer des modèles de simulation pour comprendre des phénomènes encore plus complexes.
