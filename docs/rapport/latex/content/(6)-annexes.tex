\pagenumbering{arabic}
\setcounter{page}{1}
\renewcommand{\thepage}{\hspace*{1em}A\arabic{page}}

\appendix
\phantomsection
\addcontentsline{toc}{section}{Annexes}

\section{Design de l'interface graphique}
\label{sec:annexe1}
\image{assets/images/Proposition_Interface_Graphique_feu_de_foret.png}{Design Figma de l’interface graphique}{designFigma}{1}
\newpage
\section{Interface graphique implementé}
\label{sec:annexe2}
\image{assets/images/Interface_Graphique_Implémentée.jpg}{Interface graphique réellement implémenté}{interfaceRélle}{0.80}
\newpage
\section{Algorithme de la boucle de simulation}
\label{sec:annexe3}

\begin{figure}[htbp]
    \centering
    \begin{lstlisting}
    public double[] simulation(String direction_vent, String saison){
        double continuerSimulation = 0;
        double nombreTotalArbre = 0;
        double nombreArbreEnCendre = 0;

        for(int i = 0; i < this.nb_lignes; i++) {
            for( int j = 0; j < this.nb_colonnes; j++) {
                Cellule cellule = this.getCellule(i, j);
                cellule.calculeEtatFutur(direction_vent, saison);
                if(cellule.getEtatFutur() == 2) continuerSimulation = 1;
                if(cellule.getEtat() == 1 || cellule.getEtat() == 2 || cellule.getEtat() == 3) nombreTotalArbre += 1;
                if(cellule.getEtat() == 3 || cellule.getEtatFutur() == 3) nombreArbreEnCendre += 1;
            }
        }

        return new double[]{continuerSimulation, ((nombreArbreEnCendre/nombreTotalArbre)*100)};
    }
    \end{lstlisting}
    \caption{Code de la méthode \texttt{Grille.simulation()}}
    \label{fig:boucleSimulation}
\end{figure}
\newpage
\section{Algorithme de calcul de l’état futur d’une cellule}
\label{sec:annexe4}

\begin{figure}[htbp]
    \centering
    \begin{lstlisting}
    @Override
    public void calculeEtatFutur(String direction_vent, String saison){
        // p : probabilité de propagation du feu vers la cellule
        double p;

        if(this.etat == 3) this.etat_futur = etat;
        else if(this.etat == 2) this.etat_futur = 3;
        else if(this.etat == 1){
            if(this.nombreDeVoisinsEnFeu() == 0) this.etat_futur = this.etat;
            else{
                p = this.calculeP(direction_vent, saison);
                if(this.feuVaSePropager(p)) this.etat_futur = 2;
                else this.etat_futur = etat;
            }
        }
    }
    \end{lstlisting}
    \caption{Code de la méthode \texttt{Arbre.calculeEtatFutur()}}
    \label{fig:calculEtatFutur}
\end{figure}
\newpage
\section{Algorithme de calcul de la probabilité de propagation}
\label{sec:annexe5}

\begin{figure}[htbp]
    \centering
    \begin{lstlisting}
    public double calculeP(String direction_vent, String saison){
        // Impact du voisinage sur p.
        double p = 0.6 + this.nombreDeVoisinsEnFeu() * 0.05;
        // Impact de la direction du vent sur p.
        if(!this.feuVersCellule(direction_vent) && direction_vent != null) p -= 0.1;
        if(this.feuVersCellule(direction_vent)) p += 0.5;
        // Impact de la saison sur p
        if(saison == null){}
        else if(saison.equals("HIVER")) p -= 0.1;
        else if(saison.equals("ETE")) p += 0.2;
        // Impact de l'humidité du sol sur p.
        if(this.solEstHumide()) p -= 0.1;
        // Impact du type de végétation sur p.
        if(this.arbreEstPeuInflammable()) p -= 0.2;
        // La probabilité doit être comprise en 0 et 1.
        if(p < 0) p = 0;
        if(p > 1) p = 1;
        
        return p;
    }
    \end{lstlisting}
    \caption{Code de la méthode \texttt{Arbre.calculeP()}}
    \label{fig:calculProbaPropagation}
\end{figure}
