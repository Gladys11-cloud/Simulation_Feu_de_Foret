% Définition de la classe et des paramètres principaux
\documentclass[12pt, letterpaper, table, xcdraw]{model}

% Utilisation du package pour importer des fichiers
\usepackage{import}
% Importation de styles
\usepackage{styles}
\usepackage{subcaption}
% Modifier les styles d'index et de liste
\renewcommand{\cftdotsep}{4}
\renewcommand{\cftfigpresnum}{\bfseries Figure }
\cftsetindents{figure}{0em}{5.5em}
\renewcommand{\cfttabpresnum}{\bfseries Tableau }
\cftsetindents{table}{0em}{5.5em}
% L'indentation de 1 cm est définie
\setlength{\parindent}{1cm}

\usepackage{listings}

\lstset{
    language=Java,
    basicstyle=\ttfamily\small,
    keywordstyle=\bfseries\color{blue},
    commentstyle=\itshape\color{green!40!black},
    stringstyle=\color{orange},
    showstringspaces=false,
    breaklines=true,
    captionpos=b
}

\begin{document}

  % Couverture
  \import{common/}{couverture.tex}
  \clearpage

  % Sommaire
  \import{content/}{(1)-sommaire.tex}
  \clearpage

  \import{content/}{(2)-figures_et_tables.tex}
  \clearpage
  
  \import{content/}{(3)-introduction.tex}
  \clearpage

  % Contenu
  \import{content/sections/}{1-Revue_littérature.tex}
  \clearpage

  \import{content/sections/}{2-concepts_théoriques.tex}
  \clearpage

  \import{content/sections/}{3-cas_d_étude.tex}
  \clearpage

  % Conclusion
  \import{content/}{(4)-conclusion.tex}
  \clearpage
  
  \import{content/}{(5)-bibliographie.tex}
  \clearpage
  
  \import{content/}{(6)-annexes.tex}
  \clearpage

\end{document}
